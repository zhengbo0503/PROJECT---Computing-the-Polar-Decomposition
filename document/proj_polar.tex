\documentclass[12pt]{article}

\usepackage{a4} 
%%% Algorithm %%%
\usepackage{algorithm}
\usepackage{algpseudocode}
%%%
\usepackage{color,bold-extra,mathrsfs,float,bbm}
\usepackage[tableposition=t]{caption}
\captionsetup{textfont={small,it},labelfont={small, sc}}
\captionsetup[figure]{name=Fig.}
\usepackage{comment,graphics,aliascnt}
\usepackage[dvipsnames]{xcolor}
\usepackage[colorlinks=true,
linkcolor=red,
urlcolor=red,
citecolor=blue,
bookmarksopen=true]{hyperref}
\usepackage[hyperpageref]{backref}
\renewcommand*{\backref}[1]{}% for backref < 1.33 necessary
\renewcommand*{\backrefalt}[4]{%
\ifcase #1 %
No citations.%
\or
(Cited on page #2.)
\else
(Cited on pages #2.)
\fi
}
\usepackage{enumerate,amsmath,amsfonts,amssymb,comment,mathtools}
\usepackage{amsthm}

%% Fonts %%

\usepackage[T1]{fontenc}
\DeclareFontFamily{OT1}{rsfs}{}
\DeclareFontShape{OT1}{rsfs}{n}{it}{<-> rsfs10}{}
\DeclareMathAlphabet{\mathscr}{OT1}{rsfs}{n}{it}
\usepackage{lmodern} \normalfont %to load T1lmr.fd
\DeclareFontShape{T1}{lmr}{bx}{sc} { <-> ssub * cmr/bx/sc }{}
\usefont{T1}{qzc}{m}{it}
\usepackage{eucal}

%% MATLAB Code %%% From Dr. Chris Johnson
\usepackage{listings}
\definecolor{codegreen}{rgb}{0,0.6,0}
\definecolor{codegray}{rgb}{0.5,0.5,0.5}
\definecolor{codepurple}{rgb}{0.58,0,0.82}
\definecolor{mygreen}{RGB}{28,172,0}
\definecolor{mylilas}{RGB}{170,55,241}
\definecolor{backcolour}{rgb}{0.95,0.95,1.92}
\lstdefinestyle{mystyle}{
	language=matlab,
  commentstyle=\color{codegreen},
  keywordstyle=\color{blue},
  numberstyle=\tiny\color{codegray},
  stringstyle=\color{codepurple},
  basicstyle=\linespread{1}\ttfamily\small,
  breakatwhitespace=false,
  breaklines=true,
  captionpos=b,
  keepspaces=true,
  numbers=left,
  numbersep=5pt,
  showspaces=false,
  showstringspaces=false,
  showtabs=false,
  tabsize=2,
  aboveskip=\medskipamount,
  frame=single,
}
\lstset{style=mystyle}
% \inline is a custom macro
\def\inline{\lstinline[basicstyle=\upshape\ttfamily]}

\DeclareCaptionLabelSeparator{separation}{:\quad}

%% Some definitions %%
\DeclarePairedDelimiter\ceil{\lceil}{\rceil}
\DeclarePairedDelimiter\floor{\lfloor}{\rfloor}
\usepackage{bm}
\DeclarePairedDelimiter\abs{\lvert}{\rvert}
\DeclarePairedDelimiter\inner{\langle}{\rangle}
% real and imaginary
\renewcommand{\Re}[1]{\mathfrak{Re}\left\{{#1}\right\}}
\renewcommand{\Im}[1]{\mathfrak{Im}\left\{{#1}\right\}}

\newcommand{\intii}{\int_{-\infty}^{\infty}}

%% Theorems %%
\newtheorem{theorem}{Theorem}[section]
\newtheorem{lemma}[theorem]{Lemma}
\newtheorem{proposition}[theorem]{Proposition}
\newtheorem{assumption}[theorem]{Assumption}
\newtheorem{corollary}[theorem]{Corollary}

\theoremstyle{definition}
\newtheorem{definition}[theorem]{Definition}
\newtheorem{remark}[theorem]{Remark}
\newtheorem{example}[theorem]{Example}
\newtheorem*{note}{Note}
\newtheorem*{question}{Question}
\newtheorem*{recall}{Recall}
\newtheorem{exercise}[theorem]{Exercise}

\numberwithin{equation}{section}

%% mathbb %%
\newcommand{\N}{\mathbb{N}} % natural numbers 
\newcommand{\Z}{\mathbb{Z}} % integers 
\newcommand{\Q}{\mathbb{Q}} % rationals
\newcommand{\R}{\mathbb{R}} % reals
\newcommand{\C}{\mathbb{C}} % complex
\newcommand{\F}{\mathbb{F}}

%% Probability %%
\renewcommand{\P}{\mathbb{P}} % probability
\DeclareMathOperator{\E}{\mathbb{E}} % expectation
\DeclareMathOperator{\V}{\mathbf{Var}} % variance

%% matrix and vectors %%
\newcommand{\n}{^n}
\newcommand{\Rn}{\R^n}
\newcommand{\mn}{^{m\times n}}
\newcommand{\nn}{^{n\times n}}
\renewcommand{\ij}{_{ij}}
\newcommand{\tp}{^T} 
\newcommand{\ctp}{^*}
\newcommand{\inv}{^{-1}}
\newcommand{\diag}{\mathrm{diag}}
\newcommand{\rank}{\mathrm{rank}}
\newcommand{\tr}{\mathrm{trace}}
\renewcommand{\det}{\mathrm{det}}
\newcommand{\range}{Range}
\renewcommand{\dim}{dim}
\renewcommand{\span}{span}
\newcommand{\eref}[1]{\eqref{#1}}

%% Symbols %%
\newcommand{\mat}{MATLAB}
\newcommand{\wh}{\widehat}
\newcommand{\wt}{\widetilde}
\newcommand{\wb}{\overline}
\newcommand{\grad}{\nabla}
\renewcommand{\div}{\nabla\cdot}
\newcommand{\curl}{\nabla\times}
\newcommand{\lap}{\varDelta}
\newcommand{\dd}{\mathrm{d}}
\newcommand{\pp}{\partial}
\def\eu{\mathrm{e}} % euler's constant
\def\im{\mathrm{i}} % imaginary unit

%%%% Greek Letters %%%%
\renewcommand{\a}{\alpha}
\renewcommand{\l}{\lambda}
\renewcommand{\L}{\Lambda}
\newcommand{\vL}{\varLambda}
\renewcommand{\b}{\beta}
\newcommand{\p}{\phi}
\newcommand{\x}{\xi}
\newcommand{\G}{\Gamma}
\newcommand{\vG}{\varGamma}
\renewcommand{\O}{\Omega}
\newcommand{\vO}{\varOmega}
\renewcommand{\o}{\omega}
\newcommand{\e}{\epsilon}

% MSc project macros%
\def\ycite[#1#2#3#4#5]#6{\cite[$\mit{#1#2#3#4}$#5]{#6}}
\newcommand{\iter}[1]{^{(#1)}} % iteration
\newcommand{\norm}[1]{\|{#1}\|_2} % 2-norm
\newcommand{\sign}[1]{\mathrm{sign}\left({#1}\right)}
\renewcommand{\Sigma}{\varSigma} % skew sigma
\newcommand{\off}{\mathsf{off}} % off operator
\newcommand{\gnorm}[1]{\|{#1}\|}

%% Defined Proof Evironment %%

\def\proof{\par{\bf Proof}. \ignorespaces}
\def\qedsymbol{\vbox{\hrule\hbox{%
                     \vrule height1.3ex\hskip0.8ex\vrule}\hrule}}
\def\endproof{\qquad\qedsymbol\medskip\par}

\title{Project : Computing the Polar Decomposition}
\author{Zhengbo Zhou%
    \thanks{%
        School of Mathematics,
        University of Manchester,
        Manchester, M13 9PL, England
        (\texttt{zhengbo.zhou@student.manchester.ac.uk}).
    }
}
\date{September 22, 2022}

\begin{document}
\maketitle
\tableofcontents
\newpage

\section{Norms and Singular Value Decomposition}\label{sec:norms-svd}

\subsection{Vector Norms} \label{subsec:vector-norms}

Before introducing matrix norm, a brief illustration of vector norm is necessary.

\begin{definition}
  [Vector Norm] \label{def:vector-norm}
  A vector norm on $\C^n$ is a function $\gnorm{\cdot} : \C^n \to \R$ such that it satisfies the following properties
  \begin{enumerate}
    \item $\gnorm{x} \geq 0$ for all $x\in\C\n$,
    \item $\gnorm{x} = 0$ if and only if $x = 0$,
    \item $\gnorm{\lambda x} = \abs{\lambda} \gnorm{x}$ for all $\lambda \in \C$ and $x \in \C^n$,
    \item $\gnorm{x + y} \leq \gnorm{x} + \gnorm{y}$ for all $x,y\in \C^n$.
  \end{enumerate}
\end{definition}

\begin{example}
    For $x\in\C^n$, 
    \begin{equation}\notag
        \begin{aligned}
            \text{1-norm : }& \gnorm{x}_1 = \sum_{i = 1}^n \abs{x_i},\\
            \text{2-norm (Euclidean Norm) : }& \norm{x} = \left(\sum_{i = 1}^n \abs{x_i}^2\right)^{1/2} = \sqrt{x\ctp x},\\
            \text{$\infty$-norm : } & \gnorm{x}_{\infty} = \max_{1\leq i \leq n} \abs{x_i}.
        \end{aligned}
    \end{equation}
    
    These are all special cases of the $p$-norm,
    \begin{equation}\label{eq:vector-p-norm}
        \gnorm{x}_p = \left(\sum_{i = 1}^n \abs{x_i}^p\right)^{1/p},\quad q \geq 1.
    \end{equation}
\end{example}

\subsection{Matrix Norms} \label{subsec:matrix-norms}
\begin{definition}
    [Matrix Norms]
    A matrix norm is a function $\gnorm{\cdot}:\C\mn \to \R$ satisfies the analogues of the four properties in the Definition~\ref{def:vector-norm}.
\end{definition}

\begin{example}
    [Frobenius norm]
    For $A\in\C\mn$, the Frobenius norm is defined as
    \begin{equation}
        \notag
        \gnorm{A}_F = \left(\sum_{i = 1}^m \sum_{j = 1}^n \abs{a_{ij}}^2\right)^{1/2} = \left(\tr(A\ctp A)\right)^{1/2}.
    \end{equation}
\end{example}

\begin{example}
    [Subordinate matrix norm]
    The matrix norm that is induced by a vector norm is called the subordinate norm. Suppose $\gnorm{\cdot}$ is a vector norm, the corresponding subordinate matrix norm is defined as 
    \begin{equation}
        \notag 
        \gnorm{A} = \max_{\gnorm{x} = 1} \gnorm{Ax}, \quad A \in \C\mn, \quad x \in \C^n.
    \end{equation}

    Apply this definition into \eqref{eq:vector-p-norm}, we have the definition for the matrix $p$-norm 
    \begin{equation}
        \notag 
        \gnorm{A}_p = \max_{\gnorm{x}_p = 1} \gnorm{Ax}_{p}.
    \end{equation}

    The subordinate matrix norms for $1$-, $2$- and $\infty$-norms can be shown to have the following form 
    \begin{equation}
        \notag 
        \begin{aligned}
            \gnorm{A}_1 &= \max_{1\leq j \leq n} \sum_{i = 1}^m \abs{a_{ij}}, \\
            \gnorm{A}_2 &= \left(\rho(A\ctp A)\right)^{1/2} = \sigma_{\max}(A),\\
            \gnorm{A}_\infty & = \max_{1 \leq i \leq m} \sum_{j = 1}^n \abs{a_{ij}},
        \end{aligned}
    \end{equation}
    where $\rho(A)$ represents the spectral radius of the matrix $A$ which defined as the largest eigenvalue in magnitude of $A$ and $\sigma_{\max}(A)$ represents the largest singular value of $A$.
\end{example}

We say a matrix norm $\gnorm{\cdot}$ is consistent if for all $A,B\in\C\mn$, the following inequality holds whenever the product $AB$ defines 
\begin{equation}
    \notag 
    \gnorm{AB} \leq \gnorm{A}\gnorm{B}.
\end{equation}
The Frobenius norm and all subordinate norms are consistent.

\subsection{Singular Value Decomposition}\label{subsec:svd}
\begin{theorem}
    [Singular value decomposition]
    \label{thm:svd}
    If $A\in\C\mn$, $m\geq n$, then there exists two unitary matrices $U\in\C^{m\times m}$ and $V\in\C\nn$ such that 
    \begin{equation}\label{eq:svd}
        A = U\Sigma V\ctp,\quad \Sigma = \diag(\sigma_1,\dots,\sigma_p)\in\R\mn,\quad p = \min\{m,n\},
    \end{equation}
    where $\sigma_1,\dots,\sigma_p$ are all non-negative and arranged in non-ascending order. We denote~\eqref{eq:svd} as the singular value decomposition (SVD) of $A$ and $\sigma_1,\dots,\sigma_p$ are the singular values of $A$.
\end{theorem}





\section{Polar Decomposition and its Properties}\label{sec:polar-properties}

Throughout this project, we focused on $A\in\C\nn$. In complex analysis, it is known that for any $\alpha\in\C$, we can write $\alpha$ in polar form, namely $\alpha = r\eu^{\im \theta}$. The polar decomposition is its matrix analogue.

\begin{theorem}
    [Polar Decomposition~{\ycite[2008, Theorem~8.1]{2008higham-fm}}] \label{def:matrix-norm}
    Let $A\in\C\mn$ with $m \geq n$. There exists a matrix $U\in\C\mn$ with orthonormal columns and a unique Hermitian positive semidefinite matrix $H\in\C\nn$ such that $A = UH$. The matrix $H$ is given by $H = \left(A\ctp A\right)^{1/2}$. If the matrix $A$ is full rank, then $H$ is Hermitian positive definite and $U$ is uniquely determined.
\end{theorem}

\begin{proof}
    Suppose $\rank(A) = r$, then let $A$ has a SVD $A = P\Sigma_r V\ctp$. The polar decomposition of $A$ can be formed in terms of the SVD:
    \begin{equation}
        \notag 
        A = P
        \begin{bmatrix}
            I_r & 0 \\
            0 & I_{m-r,n-r}
        \end{bmatrix}
        V\ctp V 
        \begin{bmatrix}
            \Sigma_r & 0 \\
            0 & 0_{n-r,n-r}
        \end{bmatrix}
        V\ctp =: UH.
    \end{equation}
    where 
    \begin{equation}
        \notag 
        U = P
        \begin{bmatrix}
            I_r & 0 \\
            0 & I_{m-r,n-r}
        \end{bmatrix}
        V\ctp
        ,\quad 
        H = 
        V 
        \begin{bmatrix}
            \Sigma_r & 0 \\
            0 & 0_{n-r,n-r}
        \end{bmatrix}
        V\ctp.
    \end{equation}
    
    We can test the columns' orthonormality of $U$ by performing
    \begin{equation}
        \notag 
        U\ctp U = V
        \begin{bmatrix}
            I_r & 0 \\ 0 & I_{n - r, m-r}
        \end{bmatrix}
        P\ctp P
        \begin{bmatrix}
            I_r & 0 \\ 0 & I_{m-r,n-r}
        \end{bmatrix}
        V\ctp = I_{n,n}.
    \end{equation}

    The symmetry of $H$ is obvious. Notice that, $H$ and $\begin{bmatrix} \Sigma_r & 0 \\ 0 & 0_{n-r,n-r}\end{bmatrix}$ are unitarily similar, hence they share the same eigenvalues. Equivalently speaking, the eigenvalues of $H$ are the singular values of $A$. From the Theorem~\ref{thm:svd}, the singular values of $A$ are all real and non-negative, therefore $H$ is Hermitian positive semidefinite and it is uniquely determined via 
    \begin{equation}
        \left(A\ctp A\right)^{1/2} = \left(H\ctp U\ctp U H\right)^{1/2} = \left(H^2\right)^{1/2} = H.
    \end{equation}

    If $A$ is full rank, namely $r = n$, then all the singular values of $A$ are positive and consequently $H$'s eigenvalues are all positive, therefore it is Hermitian positive definite. Clearly if $H$ is nonsingular, then $U$ is uniquely determined by $U = AH\inv$.
\end{proof}

We will refer to $U$ as the unitary polar factor.

\begin{theorem}
    For $A\in\C\mn$, let $A = UH$ be its polar decomposition. $A$ is normal if and only if $U$ and $H$ are commute.
\end{theorem}

\begin{proof}
    $(\Leftarrow)$: If $U$ and $H$ are commute, then 
    \begin{equation}
        \notag
        \begin{aligned}
            AA\ctp &= \left(UH\right)\left(UH\right)\ctp\\
                & = \left(HU\right)\left(HU\right)\ctp = HUU\ctp H\ctp = H^2. \\
            A\ctp A & = H^2 = AA\ctp.
        \end{aligned}
    \end{equation}
    Hence $A$ is normal.

    $(\Rightarrow)$: If $A$ is normal
\end{proof}




\newpage 
\nocite{*}
\bibliographystyle{aomplain}
\bibliography{bib.bib}





\end{document}
