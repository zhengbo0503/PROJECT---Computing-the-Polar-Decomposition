%%% Algorithm
\usepackage{algorithm, algpseudocode, comment}
\usepackage{booktabs, bookmark}

%%% Color, Hyperref and Backref
\usepackage{hyperref}
\hypersetup{
    colorlinks=true,
    linkcolor=blue,
    urlcolor=red,
    citecolor=blue
}
\usepackage[hyperpageref]{backref}
\renewcommand*{\backref}[1]{}
\renewcommand*{\backrefalt}[4]{
    \ifcase #1 %
    No citations.%
    \or
    (Cited on page #2.)
    \else
    (Cited on pages #2.)
    \fi
}

%%% Page format required by Prof. Nick Higham
\usepackage{pre/a4}

%% Headnote and Footnotes
\usepackage{lastpage}
\usepackage{fancyhdr}
\pagestyle{fancy}
\lhead{\scshape Stochastic Calculus}
\rhead{\scshape \rightmark}
\fancyfoot[C]{Page~\thepage~of~\pageref*{LastPage}}
\setlength{\headheight}{14.49998pt}

%%% Captions and Figures
\usepackage{caption,graphics,float}
\captionsetup*{textfont={small,it},labelfont={small, sc}}
\captionsetup*[figure]{name=Fig.}


%%% Colored Box
\usepackage{tcolorbox}
\newtcolorbox{mybox}[1]{colback=white!90!black,colframe=white!40!black,fonttitle=\scshape\centering,title=#1}

%%% AMS related
\usepackage{enumerate,amsmath,amsfonts,amssymb,amsthm,mathtools}

%% Theorems 
\newtheorem{theorem}{Theorem}[section]
\newtheorem{lemma}[theorem]{Lemma}
\newtheorem{proposition}[theorem]{Proposition}
\newtheorem{assumption}[theorem]{Assumption}
\newtheorem{corollary}[theorem]{Corollary}
\theoremstyle{definition}
\newtheorem{definition}[theorem]{Definition}
\newtheorem{remark}[theorem]{Remark}
\newtheorem{example}[theorem]{Example}
\newtheorem*{note}{Note}
\newtheorem*{question}{Question}
\newtheorem*{recall}{Recall}
\newtheorem{exercise}[theorem]{Exercise}
\numberwithin{equation}{section}


%%% Fonts

%% Computer Modern Font

% \usepackage[T1]{fontenc}
% \DeclareFontFamily{OT1}{rsfs}{}
% \DeclareFontShape{OT1}{rsfs}{n}{it}{<-> rsfs10}{}
% \DeclareMathAlphabet{\mathscr}{OT1}{rsfs}{n}{it}
% \usepackage{lmodern} \normalfont %to load T1lmr.fd
% \DeclareFontShape{T1}{lmr}{bx}{sc} { <-> ssub * cmr/bx/sc }{}
% \usefont{T1}{qzc}{m}{it}

%% Lucida Bright Font

\usepackage[math-style=iso, scale=0.9]{lucimatx}
\AtBeginDocument{ % mathbb for lucimatx
    \DeclareSymbolFont{AMSb}{U}{msb}{m}{n}
    \DeclareSymbolFontAlphabet{\mathbb}{AMSb}}

%%% MATLAB Code From Dr. Chris Johnson 
\usepackage{color}
\usepackage{xcolor}
\usepackage{listings}
\definecolor{codegreen}{rgb}{0,0.6,0}
\definecolor{codegray}{rgb}{0.5,0.5,0.5}
\definecolor{codepurple}{rgb}{0.58,0,0.82}
\definecolor{mygreen}{RGB}{28,172,0}
\definecolor{mylilas}{RGB}{170,55,241}
\definecolor{backcolour}{rgb}{0.95,0.95,1.92}
\lstdefinestyle{mystyle}{
	language=matlab,
    commentstyle=\color{codegreen},
    keywordstyle=\color{blue},
    numberstyle=\tiny\color{codegray},
    stringstyle=\color{codepurple},
    basicstyle=\linespread{1}\ttfamily\small,
    breakatwhitespace=false,
    breaklines=true,
    captionpos=b,
    keepspaces=true,
    numbers=left,
    numbersep=5pt,
    showspaces=false,
    showstringspaces=false,
    showtabs=false,
    tabsize=4,
    aboveskip=\medskipamount,
    frame=single,
}
\lstset{style=mystyle}
\def\inline{\lstinline[basicstyle=\upshape\ttfamily]}


%%% Paired labels
\DeclarePairedDelimiter\ceil{\lceil}{\rceil}
\DeclarePairedDelimiter\floor{\lfloor}{\rfloor}
\DeclarePairedDelimiter\abs{\lvert}{\rvert}
\DeclarePairedDelimiter\inner{\langle}{\rangle}

%%% vector bold
\usepackage{bm}
\renewcommand{\vec}[1]{\bm{#1}}

%%% Real and Imaginary
\renewcommand{\Re}[1]{\mathfrak{Re}\left\{{#1}\right\}}
\renewcommand{\Im}[1]{\mathfrak{Im}\left\{{#1}\right\}}

%%% Integrate from ... to ...
\newcommand{\intii}{\int_{-\infty}^{\infty}}

%%% MATHBB
\newcommand{\mb}[1]{\mathbb{#1}}
\newcommand{\N}{\mb{N}}
\newcommand{\Z}{\mb{Z}} 
\newcommand{\Q}{\mb{Q}}
\newcommand{\R}{\mb{R}} 
\newcommand{\C}{\mb{C}}
\newcommand{\F}{\mb{F}}
\renewcommand{\P}{\mb{P}} % Probability
\newcommand{\E}{\mb{E}} % Expectation
\newcommand{\V}{\mb{V}} % Variance

%%% Greek Letters
% NEVER define \l for \lambda due to Polish names in BibTeX
\renewcommand{\L}{\Lambda}
\newcommand{\vL}{\varLambda}
\newcommand{\g}{\gamma}
\newcommand{\G}{\Gamma}
\newcommand{\vG}{\varGamma}
\renewcommand{\o}{\omega}
\renewcommand{\O}{\Omega}
\newcommand{\vO}{\varOmega}
\newcommand{\s}{\sigma}
\renewcommand{\S}{\Sigma}
\newcommand{\vS}{\varSigma}
\newcommand{\eps}{\varepsilon}
\newcommand{\lap}{\varDelta}

%%% Matrix Related
\newcommand{\n}{^n}
\newcommand{\Rn}{\R^n}
\newcommand{\mn}{^{m\times n}}
\newcommand{\nn}{^{n\times n}}
\newcommand{\tp}{^T} 
\newcommand{\ctp}{^*}
\newcommand{\inv}{^{-1}}
\newcommand{\diag}{\mathrm{diag}}
\newcommand{\rank}{\mathrm{rank}}
\newcommand{\tr}{\mathrm{Tr}}
\renewcommand{\det}{\mathrm{det}}
\newcommand{\range}{Range}
\renewcommand{\dim}{dim}
\renewcommand{\span}{span}

%% Norms
\newcommand{\iter}[1]{^{(#1)}} % iteration
\newcommand{\gnorm}[1]{\|{#1}\|} % general norm
\newcommand{\tnorm}[1]{\|{#1}\|_2} % 2-norm
\newcommand{\inorm}[1]{\|{#1}\|_\infty}

%%% Over the expressions
\newcommand{\wh}{\widehat}
\newcommand{\wt}{\widetilde}
\newcommand{\wb}{\overline}

%%% Calculus
\newcommand{\grad}{\nabla}
\renewcommand{\div}{\nabla\cdot}
\newcommand{\curl}{\nabla\times}
\newcommand{\dd}{\mathrm{d}}
\newcommand{\pp}{\partial}
\def\eu{\mathrm{e}} % euler's constant
\def\im{\mathrm{i}} % imaginary unit

%%% Citation
\def\ycite[#1#2#3#4#5]#6{\cite[$\mit{#1#2#3#4}$#5]{#6}}

%%% Defined Proof Environment %% For Nick's Project Only
% \def\proof{\par{\bf Proof}. \ignorespaces}
% \def\qedsymbol{\vbox{\hrule\hbox{%
%                      \vrule height1.3ex\hskip0.8ex\vrule}\hrule}}
% \def\endproof{\qquad\qedsymbol\medskip\par}

%%% MATHCAL and MATHSCR
\usepackage{mathrsfs}
\newcommand{\mc}[1]{\mathcal{#1}} % For spaces 
\newcommand{\ms}[1]{\mathscr{#1}} % For sigma-algebra

%%% Stochastic Calculus
\newcommand{\ps}{$(\Omega,\mathscr F,\P)$}
\newcommand{\ito}{It\^o}
\usepackage{bbm}
\newcommand{\indi}{\mathbbm{1}} % indicator function
\usepackage{accsupp}
\providecommand*{\napprox}{%
  \BeginAccSupp{method=hex,unicode,ActualText=2249}%
  \not\approx
  \EndAccSupp{}%
}

